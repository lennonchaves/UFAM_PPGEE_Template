% This code is based on Coppetex project
% http://coppetex.sourceforge.net/
% created by Vicente H. F. Batista, George O. Ainsworth Jr. and Paulo 
% Laranjeira da C. Lage 
%
% Author: Mauro Freitas
%

\documentclass[msc,numbers]{ufamppgee}
\usepackage{amsmath,amssymb}
\usepackage{hyperref}
\usepackage[utf8]{inputenc}

\makelosymbols
\makeloabbreviations

\begin{document}
  \title{Título da Tese}
  \foreigntitle{Thesis Title}
  \author{Nome do Autor}{Sobrenome}
  \advisor{Prof.}{Nome do Primeiro Orientador}{Sobrenome}{D.Sc.}
  \advisor{Prof.}{Nome do Segundo Orientador}{Sobrenome}{Ph.D.}
  \advisor{Prof.}{Nome do Terceiro Orientador}{Sobrenome}{D.Sc.}

  \examiner{Prof.}{Nome do Primeiro Examinador Sobrenome}{D.Sc.}
  \examiner{Prof.}{Nome do Segundo Examinador Sobrenome}{Ph.D.}
  \examiner{Prof.}{Nome do Terceiro Examinador Sobrenome}{D.Sc.}
  \examiner{Prof.}{Nome do Quarto Examinador Sobrenome}{Ph.D.}
  \department{PEE}
  \date{02}{2016}

  \keyword{Primeira palavra-chave}
  \keyword{Segunda palavra-chave}
  \keyword{Terceira palavra-chave}

  \maketitle

  \frontmatter
  \dedication{A alguém cujo valor é digno desta dedicatória.}

  \chapter*{Agradecimentos}

  Gostaria de agradecer a todos.

  \begin{abstract}

  Apresenta-se, nesta tese, é  ...

  \end{abstract}

  \begin{foreignabstract}

  In this work, we present ...

  \end{foreignabstract}

  \tableofcontents
  \listoffigures
  \listoftables
  \printlosymbols
  \printloabbreviations

  \mainmatter
  \chapter{Introdução}

  Toda abreviatura deve ser definida antes de
  utilizada.\abbrev{UFAM}{Universidade Federal do Amazonas}

  Do mesmo modo, é imprescindível definir os símbolos, tal como o
  conjunto dos números reais $\mathbb{R}$ e o conjunto vazio $\emptyset$.
  \symbl{$\mathbb{R}$}{Conjunto dos números reais}
  \symbl{$\emptyset$}{Conjunto vazio}
  Para gerar abreviações e símbolos: 
  
  makeindex -s coppe.ist -o example.lab example.abx
  
  makeindex -s coppe.ist -o example.los example.syx
  
  Para criar a ficha bibliográfica utilize o site: http://biblioteca.ufam.edu.br/

  \chapter{Revisão Bibliográfica}

  Para ilustrar a completa adesão ao estilo de citações e listagem de
  referências bibliográficas, a Tabela~\ref{tab:citation} apresenta citações de alguns dos trabalhos, utilizando o estilo numérico.

  \begin{table}[h]
  \caption{Exemplos de citações utilizando o comando padrão
    \texttt{\textbackslash cite} do \LaTeX\ e
    o comando \texttt{\textbackslash citet},
    fornecido pelo pacote \texttt{natbib}.}
  \label{tab:citation}
  \centering
  {\footnotesize
  \begin{tabular}{|c|c|c|}
    \hline
    Tipo da Publicação & \verb|\cite| & \verb|\citet|\\
    \hline
    Livro & \cite{book-example} & \citet{book-example}\\
    Artigo & \cite{article-example} & \citet{article-example}\\
    Relat\'orio & \cite{techreport-example} & \citet{techreport-example}\\
    Relat\'orio & \cite{techreport-exampleIn} & \citet{techreport-exampleIn}\\
    Anais de Congresso & \cite{inproceedings-example} &
      \citet{inproceedings-example}\\
    S\'eries & \cite{incollection-example} & \citet{incollection-example}\\
    Em Livro & \cite{inbook-example} & \citet{inbook-example}\\
    Disserta{\c c}\~ao de mestrado & \cite{mastersthesis-example} &
      \citet{mastersthesis-example}\\
    Tese de doutorado & \cite{phdthesis-example} & \citet{phdthesis-example}\\
    \hline
  \end{tabular}}
  \end{table}

  \chapter{Método Proposto}
  \chapter{Resultados e Discussões}
  \chapter{Conclusões}
  
  \backmatter
  \bibliographystyle{coppe-unsrt}
  \bibliography{example}

  \appendix
  \chapter{Algumas Demonstrações}
\end{document}
%% 
%%
%% End of file `example.tex'.
